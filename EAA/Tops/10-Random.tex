\begin{TOP}{Randomisierte Algorithmen}
\vspace*{-2\baselineskip}\begin{itemize}[itemsep=-1pt]
	\item ein randomisierter Algorithmus fällt manche Entscheidungen zufällig
	\item die Laufzeit eines solchen Algorithmus kann variieren bei der selben Eingabe
	\item bei der Analyse eines \rA{en} wollen wir das Folgende wissen:
		\begin{enumerate}
			\item die Wahrscheinlichkeit mit der die Ausgabe eine richtige Antwort ist (die Wahrscheinlichkeit einer falschen Ausgabe sollte möglichst klein sein)
			\item die erwartete Worst-Case-Laufzeit (die Durchschnittslaufzeit aller möglichen Durchläufe des Algorithmus auf der gleichen Worst-Case-Eingabe)
		\end{enumerate}
	\item die erwartete Worst-Case-Laufzeit eines \rA{en} kann in manchem Fällen besser sein, als die Laufzeit des besten bekannten deterministischen Algorithmus für das gleiche Problem
	\item es gibt randomisierte Algorithmen, für die es keine effiziente deterministische Alternative gibt
	\item manchmal ist die erwartete Laufzeit eines \rA{en} genauso gut, wie die Laufzeit eines deterministischen Algorithmus, aber der \rA{e} ist leichter zu implementieren, braucht weniger Speicher oder kann in der Praxis schneller sein
	\item häufig sind randomisierte Algorithmen einfach zu beschreiben, aber schwer zu analysieren
\end{itemize}
%TODO Algorithmus
	\loadTop{10/01-MinCut}
	\loadTop{10/02-DivideAndConquer}
	\loadTop{10/03-EnclDisc}
	\loadTop{10/04-Klassifikation}
\end{TOP}