\subtop{Klassifikation}{-1.48}
typischerweise sind zwei Arten von \rA{en} Algorithmen bedeutend:
\begin{description}
	\item[Las Vegas:]\ \\\up
		\begin{itemize}
			\item diese Algorithmen geben immer ein richtiges Ergebnis zurück, aber die Laufzeit kann zwischen den Ausführungen variieren
			\item ZPP (Zero Probabilistic Polynomial) ist die Klasse der Entscheidungsproblemen, die Las Vegas Algorithmen mit erwartet polynomiale Laufzeit haben
			\item \textbf{Beispiel:} \textit{RandomSelect, RandomQuickSort}
		\end{itemize}
	\item[Monte Carlo:] \ \\\up
		\begin{itemize}
			\item das zurückgegebene Ergebnis ist evtl. falsch mit einer sicheren Wahrscheinlichkeit
			\item die Laufzeit ist typisch deterministisch (kann aber auch variieren)
			\item für Entscheidungsprobleme gibt es verschiedene bedeutende Typen des Monte Carlo Types:
				\begin{description}
					\item[two-sided-Fehler:] beide Antworten (wahr oder falsch) können falsch sein
					\item[one-sided-Fehler, true-biased:] immer richtig, wenn es true zurückgibt
					\item[one-sided-Fehler, false-biased:] immer falsch, wenn es false zurückgibt
				\end{description}
			\item RP (Randomized Polynomial) ist die Klasse der Entscheidungsprobleme, die einen true-biased-Monte-Carlo Algorithmus mit deterministisch polynomialer Laufzeut haben, der ja zurückgibt bei jeder ``yes''-nstanz mit Wahrscheinlichkeit mindestens $\frac{1}{2}$
			\item BPP (Bounded Probabilistic Polynomial) ist die Klasse der Entscheidungsporbleme, die einen Monte-Carlo Algorithmus mit determinischtisch polinomialer Laufzeit haben, der ja in jeder ``yes''-Instanz mit Wahrscheinlichkeit mindestens $\frac{3}{4}$ und keiner ``no''-Instanz mit Wahrscheinlichkeit mindestens $\frac{3}{4}$
			\item \textbf{Beispiel:} \textit{RandomMinCut}
		\end{itemize}
\end{description}