\subtop{Randomisiertes Divide and Conquer}{-1.72}
\vspace*{-0.75\baselineskip}\subsection*{Problem: Selection}
\vspace*{-0.5\baselineskip}\begin{itemize}[itemsep=-1pt]
	\item deterministisch: in linearer Zeit gelöst
	\item randomisierte Variante wählt das Pivot zufällig
	\item wenn immer das Maximum aus $A$ genommen wird, läuft \textit{RandomSelect} für immer
	\item Analyse der Laufzeit:
		\begin{itemize}[itemsep=-1pt]
			\item $X$ ist die zufällige Variable, die Werte aus $\mathbb{N}_0$ annehmen kann
			\item der Erwartungswert von $X$, $E(X)=\sum\limits_{j=0}^{\infty}j\cdot Pr[X=j]$ mit $Pr[X=j]$ ist die Wahrscheinlichkeit, dass $X=j$
		\end{itemize}
	\item \textit{RandomSelect} berechnet das $k$-te Element einer Menge von $n$ Elementen in erwartet linearer Zeit
		\vspace*{-1.5\baselineskip}\Proof\up
			\begin{itemize}
				\item ein Pivot $m$ ist \textit{zentral}, wenn $|A_1|,|A_2| \geq \frac{|A|}{4}$
				\item die Hälfte der Pivots ist \textit{zentral}\\
				$\Rightarrow$ Wahrscheinlichkeit ein Pivot zu haben, das zentral ist: $\frac{1}{2}\\
				\Rightarrow$ die erwartete Wartezeit, bis man ein zentrales Pivot hat ist $\leq 2$
				\item Aufteilen des Algorithmus in Phasen:
					\begin{enumerate}
						\item Phase $j$ beinhaltet die Iterationen, für die die aktuelle Anzahl $n'$ begrenzt ist durch
							\begin{center}
								$n\left(\dfrac{3}{4}\right)^{j+1}< n' \leq n\left(\dfrac{3}{4}\right)^j$
							\end{center}
						\item eine Phase $j$ wird verlassen, wenn wir ein zentrales Pivot gewählt haben\\
						$\Rightarrow$ im Durchschnitt dauert eine Phase höchstens zwei Iterationen
						\setcounter{temp}{\value{enumi}}
					\end{enumerate}
			\end{itemize}
\end{itemize}
\topbreak\up\up
\begin{itemize}
	\item[]
			\begin{itemize}
				\item[]
					\begin{enumerate}
						\setcounter{enumi}{\value{temp}}
						\item jede Iteration kann in linearer Zeit durchgeführt werden
						\item $X_j$ ist die Laufzeit, die in Phase $j$ gebraucht wird und $X$ ist die Gesamtlaufzeit
						\item durch \textit{Linearität der Erwartung} erhält man:
							\begin{center}
								$E[X]=\sum\limits_{j=0}^{\infty}E[X_j]\leq 2cn \underbrace{\sum\limits_{j=0}^{\infty} \left(\dfrac{3}{4}\right)^j}_{=\frac{1}{1-\frac{3}{4}}} = 8cn$
							\end{center}
					\end{enumerate}
			\end{itemize}
\end{itemize}
\vspace*{-0.75\baselineskip}\subsection*{Problem: Sortierung (QuickSort)}
\vspace*{-0.5\baselineskip}\begin{itemize}[itemsep=-1pt]
	\item Pivot wird zufällig gewählt
	\item \textit{RandomQuickSort} sortiert eine Menge mit $n$ Elementen in erwartet $\BigO(n\log n)$
		\vspace*{-1.5\baselineskip}\Proof\up
			\begin{itemize}[itemsep=-1pt]
				\item erwartete Anzahl an Iterationen bis ein zentrales Pivot gefunden wird ist $\leq 2\\
				\Rightarrow$ die erwartete Zeit für \textit{RandomQuickSort} (ohne rekursive Aufrufe) ist linear
				\item ein Aufruf von \textit{RandomQuickSort} auf einer Menge von $n'$ Elemente ist vom Typ $j$, falls
					\begin{center}
						$n\left(\dfrac{3}{4}\right)^{j+1}< n' \leq n\left(\dfrac{3}{4}\right)^j$
					\end{center}
				\item das Aufteilen eines Typs $j$ beinhaltet zwei Probleme vom Typ $>j$
				\item die Teilprobleme eines festen Typs sind disjunkt und die Anzahl aller Teilprobleme eines Typs ist in der Summe höchstens $n$
				\item durch die Linearität der Erwartung folgt, dass die erwartete Laufzeit aller Teilprobleme des Typs $j$ in $\BigO(n)$ liegt
				\item da der Algorithmus bei $n'\leq 3$ stoppt $\Rightarrow$ es gibt keine Teilprobleme des Typs $j+1$, falls
					\begin{center}
						$n\left(\dfrac{3}{4}\right)^j\leq 3 \Rightarrow j \geq \frac{\log(\frac{n}{3})}{\log(\frac{3}{4})}\in \Theta(\log n)$
					\end{center}
			\end{itemize}
\end{itemize}