\begin{TOP}{Union-Find-Datenstruktur}
\vspace*{-2\baselineskip}
	\begin{enumerate}
		\item es wird eine endliche Menge $X$ verwendet
		\item Ziel: dynamische Menge $\mathcal{S}$ von disjunkten Teilmengen von $X$
		\item vorhandene Methoden:
			\begin{description}
				\item[\makeset(item $x$):] erstellt eine neue Menge nur mit dem Item $x$ $(\{x\})$
				\item[\find(item $x$):] gibt die Menge mit dem Item $x$ zurück
				\item[\union(set $i$, set $j$):] erstellt eine neue Menge mit den Mengen $i,j$ und löscht die beiden Mengen $i,j$
			\end{description}
		\item man kann annehmen dass $X=\{1,\dots,n\}$ mit $n \in \mathbb{N}$ ist, da man für andere Mengen jedem Item eine einzigartige Zahl zuordnen kann
		\item jede Menge hat einen \textbf{Repräsentanten}, \find~gibt diesen zurück, \union~bekommt diese als Argumente
	\end{enumerate}
	Im Folgenden betrachten wir eine Sequenz mit $m$ Operationen \makeset, \find~und \union, wobei $n$ die Anzahl an \makeset-Operationen ist.\\\ \\
	\loadTop{03/01-Array}
	\loadTop{03/02-LinkedList}
	%TODO Tree
	\loadTop{03/03-RootedTree}
	%TODO Anwendung
	\loadTop{03/04-Anwendung}
\end{TOP}