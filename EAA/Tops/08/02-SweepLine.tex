\subtop{Sweep-Line-Methode}{-1.08}
\vspace*{0.25\baselineskip}
\begin{itemize}
	\item eine \sweep~ist eine imaginäre Linie, die eine Menge von geometrischen Objekten abarbeitet (z.B.: eine vertikale Linie, die die Objekte von links nach rechts abarbeitet)
	\item verwendete Daten:
		\begin{description}
			\item[Status der \sweep:] Beziehung zwischen den Objekten im Bezug auf die aktuelle Position der \sweep
			\item[Ereigniszeitplan (event point schedule):] Sequenz von Positionen (z.B. die x-Koordinaten von links nach rechts) an denen sich der \sweep~Status ändern kann
		\end{description}
\end{itemize}
\subsection{Schneiden von Segmenten}
\begin{description}
	\item[Problem:] Gibt es in einer Menge von $n$ Liniensegmenten mindestens ein Paar von sich schneidenden Liniensegmenten?
	\item[mögliches Auftreten:] beim Übereinanderlegen von mehreren Schichten von Informationen auf einer Karte
	\item[Lösen des Problemes:] \ \\\up
		\begin{itemize}
			\item durch Testen aller $\left(\hspace*{-0.2cm}\begin{array}{c} n\\2\end{array}\hspace*{-0.2cm}\right)$ Paaren von Liniensegmenten ob sie sich schneiden\\
			$\Rightarrow \BigO(n^2)$
			\item die Linien können sich jedoch nur schneiden, falls sich ihre Projektionen auf die x-Achse schneiden
			\item Algorithmus von \textit{Shamos und Hoey} löst das Problem mit einem (verikalen) \sweep-Ansatz in $\BigO(n\log n)$
			\item erste Annahmen (einfacher):
				\begin{enumerate}
					\item kein Liniensegment ist vertikal
					\item kein Liniensegment besteht nur aus einem Punkt
					\item Liniensegmente schneiden sich nicht in einem ihrer Endpunkte
					\item höchstens zwei Liniensegmente schneiden sich in einem Punkt
				\end{enumerate}
			\item aus \textbf{1.} folgt, dass ein Liniensegment die \sweep~in höchstens einem Punkt schneidet
			\item der Status der \sweep~ist die Ordnung der Liniensegmente, die die \sweep~schneiden (entsprechend ihrer y-Koordinate des Schnittpunktes mit der \sweep):
				\begin{itemize}
					\item $s_1,s_2$ sind zwei Liniensegmente welche die vertikale Linie $l$ schneiden:\\
					$\Rightarrow s_1 <_l s_2 \Longleftrightarrow s_1$ schneidet $l$ strikt unter $s_2$
					\item Änderung des \sweep~Status:
						\begin{enumerate}
							\item \sweep~ist auf dem linken Endpunkt eines Segmentes  (dann wird ein neues Segment in die Ordnung eingefügt), \textit{oder}
							\item \sweep~ist auf dem rechten Endpunkt eines Segmentes  (dann wird das entsprechende Segment aus der Ordnung entfernt)
							\item zwei Segmente schneiden sich (die Ordnung der Segmente wird vertauscht)
						\end{enumerate}
				\end{itemize}
			\item Algorithmus stoppt, wenn zwei Segmente gefunden wurden, die sich schneiden $\Rightarrow$ der Ereigniszeitplan entspricht der Sequenz von $2n$ Endpunkten der $n$ Segmente, geordnet in nicht-abnehmender Reihenfolge in Bezug auf ihre $x$-Koordinate
			\item \textbf{Annahme:} zwei Liniensegmente schneiden sich, da es keinen Schnittpunkt mit drei Segmenten gibt $\Rightarrow$ es muss ein $x$ geben, sodass $s_1$ ist der direkte Vorgänger oder Nachfolger von $s_2$ in Bezug auf $<_x$
			\item \textbf{Idee von Shamos und Hoey:} wenn man zwei aufeinanderfolgende Segmente findet $\rightarrow$ testen, ob sie schneiden
		\end{itemize}
\end{description}
\topbreak
\up\up
\begin{description}
	\item[]\ \\\up\up
		\begin{itemize}
			\item wenn die \sweep~$l$ den linken Endpunkt $p$ eines Liniensegmentes $s$ müssen wir für ein anderes Liniensegment $s'$, das die \sweep~schneidet, entscheiden, ob $s$ die \sweep~ober- oder unterhalb von $s'$ schneidet\\
			$\Rightarrow$ kann in konstanter Zeit entschieden werden:
				\begin{enumerate}
					\item $p'_l$: linker Endpunkt von $s'$
					\item $p'_r$: rechter Endpunkt von $s'$
				\end{enumerate}
				dann gilt: $s$ schneidet $l$ strikt unter $s' \Longleftrightarrow \rechts{p'_lp}$ ist rechts von $\rechts{p'_lp'_r}$
			\item für die Implementation wird der Status der \sweep~in der Datenstruktur $T$ repräsentiert, welche die folgenden Operationen zulässt:
				\begin{description}
					\item[$T$.\insert$(s)$:] fügt ein Segment $s$ in den \sweep~Status ein
					\item[$T$.\delete$(s)$:] löscht ein Segment $s$ aus dem \sweep~Status
					\item[$T$.\pred$(s)$:] gibt das Segment zurück, das die \sweep~direkt unter $s$ schneidet
					\item[$T$.\succ$(s)$:] gibt das Segment zurück, das die \sweep~direkt über $s$ schneidet
				\end{description}
			mit balancierten binären Suchbäumen können diese Operationen in $\BigO(\log n)$ ausgeführt werden, falls es $\BigO(n)$ Elemente gibt
			\item der Algorithmus von \textit{Shamos und Hoey} kann in $\BigO(n\log n)$ ausgeführt werden:
				\begin{itemize}
					\item $2n$ Endpunkte $\Rightarrow$ können in $\BigO(n\log n)$ sortiert werden
					\item jede der $2n$ Iterationen der For-Schleife braucht eine konstante Anzahl an Suchbaum-Operationen
				\end{itemize}
		\end{itemize}
	\item[Spezialfälle:] \ \\\up
		\begin{description}
			\item[vertikale Segmente:] \ \\\up
				\begin{itemize}
					\item man kann die Richtung der \sweep~stören, sodass die \sweep~mit keinem anderen Liniensegment kollinear ist\\
						$\Rightarrow$ kann fehleranfällig sein
					\item stattdessen wird die \sweep~``virtuell gestört'' durch betrachten des tiefsten Endpunktes eines vertikalen Liniensegmentes als linken Endpunkt und den obersten als seinen rechten Endpunkt
				\end{itemize}
			\item[Punktsegmente:] zweimaliges Hinzufügen des einzelnen Punktes in den Ereigniszeitplans: einmal als linker Endpunkt und einmal als rechter Endpunkt
			\item[Schnitt im Endpunkt:] \ \\\up
				\begin{itemize}
					\item $p$ ist Endpunkt eines Segmentes $s$
					\item 1. Annahme: $p$ ist auch in einem Segment $s' = \oben{p'_lp'_r}$ enthalten
					\item 2. Annahme: $s'$ wurde in den \sweep~Status eingefügt vor dem Betrachten des Ereignispunktes $p$
					\item falls $p$ links von $s$ liegt: Beginn mit Einfügend von $s$ in den \sweep~Status
					\item notwendiger Vergleich: liegt $\rechts{p'_lp}$ rechts von $\rechts{p'_lp'_r}$\\
					$\Rightarrow \rechts{p'_lp}$ und $\rechts{p'_lp'_r}$ sind kollinear (bedeutet $s$ und $s'$ sind ``gleich'' im Bezug auf die aktuelle Ordnung)\\
					$\Rightarrow s$ und $s'$ schneiden sich (bzw. \textbf{allgemein:} Einfügen von $s$ in $T$ übereinstimmend mit der Ordnung $s \leq s'$ falls $(p-p'_l)\times (p'_r - p'_l)\geq 0$)
					\item nach Einfügen von $s$ muss $s'$ der Vorgänger oder Nachfolger von $s$ sein und wir finden einen Schnittpunkt
					\item ist $p$ der linke Endpunkt von $s$ hätte der Algorithmus schon im vorherigen Ereignispunkt einen Schnittpunkt gefunden
				\end{itemize}
			\item[mehr als zwei Segmente schneiden sich in einem Punkt:] für zwei dieser Segmente ist es schon wahr, dass sie Vorgänger und Nachfolger sind für eine geeignet \sweep~links des Schnittpunktes
		\end{description}
\end{description}
\topbreak
\up\up
\subsection{Voronoi-Diagramme}
\begin{description}
	\item[Problem:] ``Telefonzellenproblem'':
		\begin{description}
			\item[gesucht:] eine Unterteilung der Ebene in $n$ (Anzahl von Knoten) Zellen mithilfe einer Distanzfunktion $d:\mathbb{R}^2 \rightarrow \mathbb{R}_{\geq 0}$
			\item[mathematisch:] $V(p_i) = \{p \in \mathbb{R}^2; d(p,p_i) \leq d(p,p_j), j=1,\dots,n\},~~i=1,\dots,n$
		\end{description}
\end{description}
\begin{itemize}
	\item Distanzfunktion $d$ ist die euklidische Distanz:\\
		$d\left(\aoverb{x_1}{y_1},\aoverb{x_2}{y_2}\right) = \sqrt{(x_1-x_2)^2+(y_1-y_2)^2}$
	\item $V(p_1),\dots,V(p_n)$ sind \textit{Voronoi-Zellen}
	\item ein Voronoi-Diagramm $Vor(P)$ besteht aus den Grenzen der Voronoi-Zellen
	\item Knoten $V$ von $Vor(P)$ sind die Punkte, die auf den Grenzen von mindestens drei Zellen liegen
	\item Kanten $E$ von $Vor(P)$ sind die verbundenen Teilmengen von $Vor(P)$ ohne $V$
	\item manche Kanten von $Vor(P)$ können unendliche Länge haben
	\item Knoten aus $P$ bezeichnen wir als \site s
	\item ein Punkt $p$ ist ein Knoten in $Vor(P) \Longleftrightarrow p$ ist Zentrum eines Kreises mit mindestens drei \site s~auf seinem Umkreis und keiner \site~innerhalb des Kreises
	\item ein Punkt $p$ ist auf einer Kante in $Vor(P) \Longleftrightarrow p$ ist Zentrum eines Kreises mit genau zwei \site s~auf seinem Umkreis und keiner \site~innerhalb des Kreises
	\item eine Voronoi-Zelle eines Punktes $p_i,i=1,\dots,n$ kann wie folgt konstruiert werden:
		\begin{itemize}
			\item für alle $p_j,j\neq i$ teilt die Mittelsenkrechte $\oben{p_ip_j}$ die Fläche in zwei Halbebenen
			\item $H_{p_j}(p_i)$ ist die Halbfläche, die $p_i$ enthält\\
			$\Rightarrow V(p_i) = \bigcap\limits_{j\neq i} H_{p_j}(p_i)$
		\end{itemize}
	\item wenn das Voronoi-Diagramm auf diese Weise konstruiert wird, müssen die Schnittpunkte der $n-1$ Mittelsenkrechten berechnet werden, allerdings ist die Anzahl der Knoten und Kanten linear zur Anzahl der \site s
\end{itemize}
\begin{description}
	\item[Lemma:] Ein Voronoi-Diagramm einer Menge von $n\geq 3$ \site s hat höchstens $2n-5$ Knoten und $3n-6$ Kanten
	\up\Proof
	\up\begin{itemize}
		\item Annahme: alle \site s liegen auf einer Linie\\
			$\Rightarrow$ Voronoi-Diagramm besteht aus $n-1$ parallelen Linien\\
			$\Rightarrow$ $n-1$ Kanten, keine Knoten
		\item sonst ist das Diagramm verbunden und alle Kanten sind Segmente oder Halblinien
		\item zur Betrachtung des Diagramms als normalen planaren Graphen:
			\begin{itemize}
				\item Hinzufügen eines Knotens $v_{\infty}$ als künstlicher Endknoten der Halblinien
				\item beinhaltet dann einen planaren verbundenen Graphen mit $n$ Flächen, gleich vielen Kanten wie das Voronoi-Diagramm und einem Knoten mehr als das Voronoi-Diagramm
				\item mir der eulerschen Formel erhalten wir: $\#$Flächen $=f = |E|-|V|+1+k$, wobei $k$ der Anzahl der Zusammenhangskomponenten entspricht (bei einem verbundenen Graphen gilt $k=1$)
				\item jede Kante ist inzident zu zwei Knoten
				\item jeder Knoten (auch $v_{\infty}$) ist mindestens zu drei Kanten inzident\\
				$\Rightarrow 2\cdot |E| \geq 3\cdot |V| \Longrightarrow |V| \leq \dfrac{2}{3}\cdot |E| \Longrightarrow n = |E|-|V|+1+k \geq |E|-\dfrac{2}{3}\cdot |E|+2\\
				\Longrightarrow |E|\leq 3n -6 \Longrightarrow |V|\leq \dfrac{2}{3}|E| \leq 2n -4$
			\end{itemize}
	\end{itemize}
\end{description}
\topbreak
\up\up\up
\begin{description}
	\item[]\ \\\up
		\begin{itemize}
			\item da $V$ den fiktiven Knoten $v_{\infty}$ enthält, wissen wir jetzt, dass ein Voronoi-Diagramm höchstens $|V|-1 \leq 2n-5$ Knoten enthält
		\end{itemize}
\end{description}
\begin{itemize}
	\item \begin{description}
			\item[Algorithmus von Fortune:] zur Einfachheit nehmen wir an, dass es keinen Kreis gibt mit vier \site s auf seinem Umkreis und kein \site~in seinem Inneren \\\up
				\begin{description}
					\item[\sweep~Status:] \ \\\up
						\begin{itemize}
							\item jeder Punkt, der näher zu einem Punkt links der \sweep~$l$ ist als zu $l$ selbst, kann nicht in einer Voronoi-Zelle einer \site~sein, die rechts der \sweep~liegt
							\item die \textbf{\beach}~ist eine Kurve, die die Menge der Punkte, die näher zu einem Punkt links von $l$ als zu $l$ selbst sind, von den Punkten, die näher an $l$ sind, trennt
							\item die \beach~ist \textbf{y-monoton} (d.h. jede horizontale Linie schneidet die \beach~in genau einem Punkt)
							\item wenn es genau eine \site~links der \sweep~gibt, dann ist die \beach~eine (gedrehte) Parabel
							\item allgemein: die \beach~ist eine Sequenz von Parabelbögen, wobei jeder Bogen zu einer \site~gehört
							\item manche Parabeln können mehrere Teile zu Teilstrecken der \beach~beisteuern
							\item ein Schnittpunkt zwischen zwei aufeinanderfolgenden Parabelbögen auf der \beach~wird \textbf{\bpoint} genannt
							\item der Status der \sweep~ist die geordnete Sequenz von Parabelbögen und \bpoint s auf der \beach
							\item Speicherung des \sweep~Status in einem binären Suchbaum
						\end{itemize}
						\example{\beach~\&~\text{binärer Suchbaum}}{\ \\\up
						\begin{minipage}{0.2\textwidth}
							\includegraphics[scale=0.25]{Pics/8_beachline-tree.png}
						\end{minipage}
						\hfill
						\begin{minipage}{0.75\textwidth}
						Baum vor dem Einfügen von $p_6$:\\
						\resizebox{0.7\textwidth}{!}{
							\input{Pics/8_treeBefore.pgf}}
						\end{minipage}\\\\\\
						\begin{minipage}{0.4\textwidth}
						Baum nach dem Einfügen von $p_6$:
						\end{minipage}
						\hfill
						\begin{minipage}{0.5\textwidth}
						Baum nach der Neuausrichtung:
						\end{minipage}\\
						\begin{minipage}{0.4\textwidth}
						\resizebox{\textwidth}{!}{
							\input{Pics/8_treeAfter.pgf}}
						\end{minipage}
						\hfill
						\begin{minipage}{0.5\textwidth}
						\resizebox{0.8\textwidth}{!}{
							\input{Pics/8_treeRebal.pgf}}
						\end{minipage}\\\ \\
						}
				\end{description}
		\end{description}
\end{itemize}
\topbreak
\up\up
\begin{itemize}
	\item[] \begin{description}
			\item[]\ \\\up\up \begin{description}
					\item[Ereigniszeitplan:]\ \\\up
						\begin{itemize}
							\item der Status der \sweep~ändert sich, wenn ein neuer Parabelbogen auf der \beach~auftaucht bzw. wenn ein Parabelbogen von der \beach~verschwindet
							\item verwendete Datenstruktur: \PQ
						\end{itemize}
						\begin{description}
							\item[\site-Event:]\ \\\up
								\begin{minipage}{0.5\textwidth}
									\begin{itemize}
										\item ein neuer Parabelbogen kann \textbf{nur} dann auf der \beach~auftauchen, wenn die \sweep~einen \site~$p$ enthält
										\item somit sind alle \site s, sortiert in nicht-absteigender Reihenfolge ihrer $x$-Koordinate eine Teil-Sequenz des Ereigniszeitplans
										\item Einfügen eines neuen Parabelbogens in den Suchbaum:
									\end{itemize}
								\end{minipage}\hfill
								\begin{minipage}{0.3\textwidth}
									\includegraphics[width=\textwidth]{Pics/8_siteevent.png}
								\end{minipage}\\
								\begin{itemize}
									\item[] \begin{enumerate}
											\item traversieren des Suchbaums
											\item am Ende wird die $y$-Koordinate von $p$ mit der Koordinate der Positionen der aktuellen \bpoint~verglichen
											\item die Suche endet im Parabelbogen $\alpha$ links von $p$
										\end{enumerate}
									\item $q$ ist das Label von $\alpha$
									\item $\alpha$ ist ein Fragment der Parabel, welches die Punkte näher bei der \site~$q$ von den Punkten, die näher bei der \sweep~liegen trennt
									\item Ersetzten von $\alpha$ in Suchbaum durch einen Teilbaum, der die Sequenz $q,(q,p),p(p,q),q$ von drei Parabelbögen und ihren \bpoint s repräsentiert (Parabelbögen sind die Blätter)
									\item $p$ kann direkt rechts von einem \bpoint~liegen:
										\begin{enumerate}
											\item Aufteilen von $\alpha$ in zwei Teile, wobei ein Teil die Länge $0$ hat
											\item dieser Bogen mit Länge $0$ wird sofort als verschwindender Bogen betrachtet
										\end{enumerate}
								\end{itemize}
							\item[\kreis:]\ \\\up
								\begin{minipage}{0.5\textwidth}
									\begin{itemize}
										\item ein Parabelbogen $\alpha$ mit Label $q$ verschwindet dann aus dem \sweep~Status, wenn der \bpoint~$(p_1,q)$ unter $\alpha$ mit dem \bpoint~$(q,p_2)$ über $\alpha$ übereinstimmt\\
										$\Rightarrow$ der Punkt $p_0$, an dem $\alpha$ verschwindet, ist der Punkt auf der \beach, der gleich weit entfernt ist von den \site s $p_1,q,p_2$, wie zu der \sweep\\
										$\Rightarrow p_0$ ist das Zentrum eines Kreises, der die \sweep~berührt, die \site s $p_1,q,p_2$ sind auf seinem Umkreis und kein \site~liegt innerhalb
									\end{itemize}
								\end{minipage}\hfill
								\begin{minipage}{0.3\textwidth}
									\includegraphics[width=\textwidth]{Pics/8_circleevent.png}
								\end{minipage}\\
								\begin{itemize}
									\item ein Parabelbogen $\alpha$ wird nicht verschwinden, falls ein Parabelbogen direkt über und unter $\alpha$ die gleiche \site~als Label haben
									\item ein Punkt ist der Ort, an dem ein Parabelbogen verschwindet\\ $\Longleftrightarrow$ der Punkt ist ein Knoten im Voronoi Diagramm
									\item um ein \kreis~in den Ereigniszeitplan einzubinden tun wir folgendes:
										\begin{enumerate}
											\item an jedem Ereignispunkt gilt: die drei Parabelbögen mit den Labeln $p_1,p_2,p_3$ erscheinen neu auf der \beach
											\item bei einem \site~liegt einer der drei oben genannten Bögen auf der \sweep, bei einem \kreis~ist gerade ein Bogen verschwunden bei $(p_1,p_2)$ oder bei $(p_2,p_3)$
											\setcounter{temp}{\value{enumi}}
										\end{enumerate}
								\end{itemize}
						\end{description}
				\end{description}
		\end{description}
\end{itemize}
\topbreak
\up\up\up\up
\begin{itemize}
	\item[] \begin{description}
			\item[]\ \\\up\up \begin{description}
					\item[]\ \\\up
						\begin{description}
							\item[]\ \\\up
								\begin{minipage}{0.84\textwidth}
									\begin{itemize}
										\item[]
											\begin{enumerate}
											\setcounter{enumi}{\value{temp}}
												\item zwei \bpoint s $(p_1,p_2)$ und $(p_2,p_3)$ konvergieren, wenn
													\begin{enumerate}
														\item $p_1\neq p_3$,
														\item $p_1,p_2,p_3$ liegen nicht auf einer Linie und
														\item der rechteste Punkt $r$ eines eindeutigen Kreises durch $p_1,p_2,p_3$ ist rechts der aktuellen \sweep, oder $r$ ist die \site, die gerade rechts von einem \bpoint~eingefügt wurde
													\end{enumerate}
											\end{enumerate}
										\item wenn zwei \bpoint s konvergieren, wird ein ``potentieller'' \kreis~in den Ereigniszeitplan eingefügt
										\vspace*{-0.5\baselineskip}
										\item ein \kreis~bei $r$ kann ein falscher Alarm gewesen sein (kann passieren, wenn die Bögen $p_1$ oder $p_3$ nicht vor $p_2$ durch ein \kreis~verschwindet, oder weil $p_1,p_2$ oder $p_3$ in zwei Teile durch ein \site-Event aufgeteilt wurde)
										\vspace*{-0.5\baselineskip}
										\item im letzten Fall wird der der \kreis~wieder aus dem Ereigniszeitplan entfernt
										\vspace*{-0.5\baselineskip}
										\item hierfür speichern werden die Zeiger der Parabelbögen zu den \kreis~gespeichert, in denen sie beteiligt waren
										\vspace*{-0.5\baselineskip}
										\item zum Löschen eines verschwundenen Parabelbogens $\alpha$ aus dem Suchbaum, wird ein Zeiger vom \kreis~zu $\alpha$ verwendet:
											\begin{enumerate}
												\item Löschen von $\alpha$
												\item falls $\alpha$ nicht beides hat (Vorgänger und Nachfolger) im \sweep~Status (beides wären \bpoint, falls es sie gibt): Löschen des bestehenden (Vorgänger oder Nachfolger) aus dem Suchbaum
												\item sonst: Löschen des Vorgängers von $\alpha$ und Ersetzen des Nachfolgers von $\alpha$ mit dem \bpoint~zwischen dem Parabelbogen direkt unter $\alpha$ und dem Parabelbögen direkt über $\alpha$
												\item die einzige Änderung im Suchbaum während einer Löschoperation ist, dass das gelöschte Element durch seinen Nachfolger ersetzt wird
												\item da $\alpha$ schon gelöscht wurde, kann höchstens der Vorgänger von $\alpha$ durch den Nachfolger von $\alpha$ ersetzt werden
												\item Eigenschaft der inneren Knoten des Suchbaums (die \bpoint s) bleibt erhalten
											\end{enumerate}
									\end{itemize}
								\end{minipage}\\\\
						\end{description}
				\end{description}
		\end{description}
	\item \begin{description}
			\item[Konstruktion des Voronoi-Diagrammes:]\ \\\up\up
				\begin{itemize}
					\item Algorithmus mit der \sweep~basiert auf:\vspace*{-0.5\baselineskip}
						\begin{enumerate}
							\item ein Punkt ist ein Knoten in einem Voronoi-Diagramm $\Longleftrightarrow$ der Punkt ist das Zentrum eines \kreis
							\vspace*{-0.5\baselineskip}
							\item die \bpoint s stecken die Kanten des Voronoi-Diagrammes ab
						\end{enumerate}
					\up\item Speicherung des Voronoi-Diagrammes: doppelt-verkettete Kantenliste verwendet
					\vspace*{-0.5\baselineskip}\item immer, wenn ein neuer \bpoint~in die \beach~eingefügt wird, erstellen wir zwei neue Kanten $e$ und $Twin(e)$, mit den zugehörigen \site s der \bpoint s als ihre inzidenten Flächen
					\vspace*{-0.5\baselineskip}\item immer, wenn ein \kreis~behandelt wird, wird ein neuer Knoten $v$ erstellt
					\vspace*{-0.5\baselineskip}\item bei jedem \kreis~verschwinden zwei \bpoint s und ein neuer \bpoint~entsteht
				\end{itemize}\up
				\begin{minipage}{0.3\textwidth}
					\input{Pics/8_edgelist.pgf}
				\end{minipage}\hfill
				\begin{minipage}{0.6\textwidth}
					\vspace*{-3\baselineskip}\begin{itemize}
						\item die Halbkanten werden mit den \bpoint s zu $v$ und zu der anderen Halbkante verlinkt
						\vspace*{-0.5\baselineskip}\item nachdem die \sweep~alle Punkte durchlaufen hat, wurde eine Box berechnet, die alle \site s und alle Knoten des Voronoi-Diagrammes beinhaltet, sowie alle Halbkanten, die den Halblinien entsprechen, die in der Box verankert sind
					\end{itemize}
				\end{minipage}
		\end{description}
\end{itemize}
\topbreak
\up\up
\begin{itemize}
	\item[]\begin{itemize}
			\item \example{\text{Bestimmung der einzelnen Kanten für eine Kante }e_{1,2}}{
				\input{Pics/8_edgelist-circle.pgf}
			}
		\end{itemize}
	\item \textbf{Laufzeit:}
		\begin{itemize}
			\item Ereignispunkte werden zugehörig zu einer \site~oder einem Knoten des Voronoi-Diagrammes behandelt
			\item bei $n$ \site s gibt es \textbf{höchstens} $3n-5$ Schritte des \sweep~Algorithmus
			\item ein \site-Event erhöht die Anzahl an Parabelbögen und \bpoint s auf der \beach~ um höchstens $4$
			\item ein \kreis~erhöht die Anzahl auf der \beach~nicht
			\item auf der \beach~sind höchstens $4n$ Elemente
			\item bei jedem Schritt des Algorithmus benötigt man eine konstante Anzahl an Suchbaum- und \PQ-Operationen, die jede in logarithmischer Zeit erfolgen
			\item Gesamtlaufzeit: $\BigO(n\log n)$
		\end{itemize}
\end{itemize}
\subsection{Anwendung: Minimale Spannbäume}
\begin{itemize}
	\item benutzen eines beliebigen MST-Algorithmus (mit $\Theta(n^2)$ Kanten)
	\vspace*{-.5\baselineskip}\item brauchen Graphen mit weniger Kanten aber allen MSTs: \textbf{\dg/Triangulation} mit den Knoten $P$
	\vspace*{-.5\baselineskip}\item der \dg~ist der Dual-Graph des Voronoi-Diagrammes von $P$
	\vspace*{-.5\baselineskip}\item ein MST von $G=\left(P, \aoverb{V}{2}\right)$ ist ein Teilgraph des \dg:\vspace*{-1.5\baselineskip}
		\Proof\vspace*{-.5\baselineskip}
			\begin{itemize}
				\item $\{p_1,p_2\}$ ist eine Kante eines MST in $G$
				\item betrachten des kleinsten Kreises, der $p_1,p_2$ enthält (der Kreis mit Durchmesser $\oben{p_1p_2}$)
				\item wenn es einen Punkt $p$ auf diesem Kreis gibt gilt:
				\begin{center}
					$d(p_1,p_2)>d(p_1,p)\text{ und } d(p_1,p_2)>d(p_2,p)$
				\end{center}
				\item $p_1,p,p_2$ ist ein Kreis in $G$ und eine Kante, die das größte Gewicht in einem Kreis hat, ist niemals Teil eines MST (rote Regel)\\
				$\Rightarrow$ es gibt keinen Punkt $p$
				\item somit gibt es nur einen Kreis mit $p_1,p_2$ ohne einen Punkt aus $P$ innerhalb oder auf dem Kreis
				\item da die Voronoi-Zellen von $p_1,p_2$ durch eine Kante getrennt sind\\
				$\Rightarrow \{p_1,p_2\}$ ist eine Kante des \dg
			\end{itemize}
	\vspace*{-.5\baselineskip}\item ein EMST mit $n$ Punkten in der Ebene, kann in $\BigO(n\log n)$ berechnet werden\up
		\Proof\vspace*{-.5\baselineskip}
			\begin{itemize}
				\item Voronoi-Diagramm kann in $\BigO(n\log n)$ berechnet werden
				\item der \dg~kann in $\BigO(n)$ berechnet werden
				\item der \dg~hat $n$ Knoten und höchstens $3n-6$ Kanten
				\item mit Kruskal's Algorithmus kann der MST in $\BigO(n\log n)$ berechnet werden
			\end{itemize}
\end{itemize}