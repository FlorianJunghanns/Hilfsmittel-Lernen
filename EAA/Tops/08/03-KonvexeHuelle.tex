\subtop{Konvexe Hülle}{-1.09}
\begin{itemize}[itemsep=0pt]
	\item ein Polygon ist gegeben durch eine Sequenz von Punkten $\langle p_1,\dots,p_n\rangle$ mit den Seiten $\oben{p_1,p_2},\dots,\oben{p_{n-1}p_n},\oben{p_np_1}$
	\item einfaches Polygon: keine zwei Seiten schneiden sich 
	\item eine Menge $S\subseteq \mathbb{R}^2$ ist \textbf{konvex}, falls $\oben{q_1q_2} \subseteq S$ für alle Punkte $p_1,p_2 \in S$
	\item ein Polygon ist konvex, wenn die Vereinigung seiner Begrenzung und seinem Inneren konvex ist
	\item ein Polygon ist konvex $\Longleftrightarrow$ der Innenwinkel ist höchstens $\pi$
	\item eine konvexe Hülle $H(Q)$ einer Menge $Q$ von Punkten ist ein konvexes Polygon mit der minimalen Anzahl an Knoten, sodass
		\begin{enumerate}
			\item die Knoten aus $H(Q)$ sind eine Teilmenge von $Q$
			\item $Q$ ist enthalten im Inneren oder auf der Begrenzung von $H(Q)$
		\end{enumerate}
	\item die Vereinigung des Inneren und der Begrenzung der konvexen Hülle einer Menge $Q$ ist der Schnitt aller konvexen Mengen, die $Q$ enthalten
	\item Knoten und Kanten, die inzident zur Außenfläche des \dg~von $Q$ sind, bilden die konvexe Hülle
	\item eine konvexe Hülle kann in $\BigO(n\log n)$ berechnet werden
\end{itemize}

\subsection{Untere Laufzeitschranke}
\begin{itemize}[itemsep=0pt]
	\item in einem algebraischen Entscheidungsbaum-Modell der $d$-ten Ordnung kann gefragt werden, ob Polynome des Grades höchstens $d$ positiv, null oder negativ bei der Eingabe\\
		$\Rightarrow$ Worst-Case-Laufzeit des Sortierens ist immer noch in $\Omega(n\log n)$
	\item betrachten für eine Menge $X=\{x_1,\dots,x_n\}$ von reellen Zahlen die Menge $Q=\{(x_1,x_1^2),\dots,(x_n,x_n^2)\}$ von Punkten in der Ebene\\
		$\Rightarrow $ alle Punkte aus $Q$ sind in der konvexen Hülle enthalten
	\item $X$ wird sortiert, indem man $H(Q)$ entgegen dem Uhrzeigersinn abarbeitet
	\item $T(n)$ ist die Laufzeit zum Berechnen der konvexen Hülle\\
		$\Rightarrow$ Sortierung kann in $\BigO(T(n)+n)$ erfolgen
	\item Worst-Case-Laufzeit zum berechnen der konvexen Hülle ist $\Omega(n\log n)$
	\item durch das Voronoi-Diagramm und den \dg~können wir die konvexe Hülle in asymptotisch optimaler Zeit konstruieren
	\item dieser Ansatz ist komplizierter als der daraus ziehbare Nutzen
	\item folgende Algorithmen basieren auf einer rotierenden \sweep
\end{itemize}
\subsection{Graham's Scan}
\begin{itemize}[itemsep=0pt]
	\item $Q$ ist die Menge von $n$ Punkten in der Ebene$q_0\in Q$
	\item $q_0\in Q$ ist der am weitesten links liegende unterste Punkt aus $Q$
	\item die Punkte werden mithilfe eines Strahls, ausgehend aus $q_0$ und rotierend entgegen des Uhrzeigersinnes, abgearbeitet
	\item der Algorithmus arbeitet die Punkte $q\in Q\setminus \{q_0\}$ in ansteigender Ordnung der Winkel zwischen $\rechts{q_0q}$ und der horizontalen Linie durch $q_0$ in positiver Richtung
	\begin{center}(in Bezug auf die Ordnung $q<_{q_0}q' \Longleftrightarrow \rechts{q_0q}$ liegt rechts von $\rechts{q_0q'}$)\end{center}
	\item wenn es eine Linie durch $q_0$ und mehrere Punkte aus $Q\setminus \{q_0\}$ gibt, wird nur der am weitesten entfernte Punkt betrachtet (nur dieser kann ein Punkt von $H(Q)$ sein)
	\item als Vorverarbeitung kann man alle Punkte außer den am weitesten entfernten Punkt löschen (bei mehreren Punkten auf einer Linie)
	\item iteratives Berechnen der konvexen Hülle von $\{q_0,\dots,q_i\},~i=3,\dots,m$
\end{itemize}
\topbreak
\up\up
\begin{itemize}[itemsep=0pt]
	\item durch die Vorverarbeitung ist $\langle q_0,q_1,q_2\rangle$ die konvexe Hülle von $\{q_0,q_1,q_2\}$
	\item $S=\langle q_0=p_1,p_2,\dots,p_{|S|}\rangle$ ist die konvexe Hülle von $\{q_0,\dots,q_{i-1}\}$
	\item beim Hinzufügen von $q_i$ muss der Innenwinkel geprüft werden (müssen wir in dem Polygon $S+q_i=\langle p_1,\dots,p_{|S|},q_i\rangle$ an der Stelle $p_{|S|}$ auf dem Weg von $q_i$ nach $p_{|S|-1}$ rechts abbiegen)\\
	ist der Innenwinkel größer als $\pi$, ist $p_{|S|}$ nicht Teil der konvexen Hülle und wird entfernt und wird nicht wieder als potentieller Teil der konvexen Hülle betrachtet
	\item iteratives Wiederholen des letzten Schrittes bis der Innenwinkel wieder kleiner als $\pi$ ist
	\item hieraus erhalten wir ein Polygon $\langle q_0=p_1,p_2,\dots,p_j,q_i\rangle$, wobei der Innenwinkel bei $p_2,\dots,p_{j-1} <\pi$, weil es schon ein konvexes Polygon war
	\item somit liegt $q_i$ links von
		\begin{enumerate}
			\item $\rechts{p_{j-1}p_j}$ durch Konstruktion
			\item $\rechts{p_{1}p_j}$ durch die Ordnung
			\item $\rechts{p_{1}p_2}$ durch die Wahl $p_1=q_0$
		\end{enumerate}
	\item \textbf{Laufzeit:} (\algo{Graham's Scan}{arg2})
		\begin{enumerate}
			\item Sortieren in $\BigO(n\log n)$\\
				$\Rightarrow$ alle Punkte von $Q$, die im Inneren eines Segmentes zwischen $q_0$ und einem Punkt $q\in Q$ liegen, sind direkt nach $q$ einsortiert
			\item Löschen der Punkte in (gesamt betrachtet) linearer Zeit
			\item alle übrigen Punkte werden einmal auf den Stack gepushed und höchstens einmal gepopped
			\item die for-Schleife liegt in linearer Zeit
			\item somit wird die Laufzeit vom Sortiervorgang dominiert und der Algorithmus liegt in $\BigO(n\log n)$
		\end{enumerate}
\end{itemize}
\subsection{Jarvis' march}
\begin{itemize}[itemsep=0pt]
	\item um die $\Omega(n\log n)$ Laufzeit zur Berechnung einer konvexen Hülle von $n$ Punkten zu erhalten, wird benutzt, dass alle Eingabepunkte auf der konvexen Hülle liegen
	\item ist besser, als die $\Omega(n\log n)$-Grenze, wenn nur wenige Punkte auf der konvexen Hülle liegen
	\item Start: am weitesten links liegender unterster Punkt
	\item der analoge Algorithmus in 3D heißt ``gift wrapping algorithm''
	\item die ersten $k$ Knoten der konvexen Hülle ($q_0=p_1,\dots,p_k$) sind schon entgegen des Uhrzeigersinnes berechnet worden\\
	$\Rightarrow p_{k+1}$ ist der nächste Punkt, den man besucht, wenn man kreisförmig um $p_k$ scannt (Start bei $\oben{p_kp_{k-1}}$)\\
	wenn es mehrere dieser nächsten Punkte gibt, dann ist $p_{k+1}$ der am weitesten von $p_k$ entfernte Punkt
	\item da $p_k$ ein Punkt der konvexen Hülle war, weiß man, dass der Winkel entgegen dem Uhrzeigersinn zwischen $\oben{p_kp_{k-1}}$ und $\oben{p_kq},~q\in Q\setminus \{p_k\}$ größer als $\pi$ ist\\
	$\Rightarrow p_{k+1}$ ist der minimale Punkt aus $Q\setminus \{p_k\}$ im Bezug auf die Ordnung $<_{p_k}$
	\item \textbf{Laufzeit:} (\algo{Jarvis' march}{\begin{algorithm}[H]
	\SetAlgoVlined
	\SetKwProg{Fn}{Function}{}{end}
	\KwIn{Menge $Q$ von $n$ Punkten in $\mathbb{R}^2$}
	\KwOut{Liste $L$ mit Knoten der konvexen Hülle $H(Q)$ im Uhrzeigersinn geordnet}
	\BlankLine
	\Begin{
		$q_0 \leftarrow$ der am weitesten links liegende der untersten Punkte in $Q$\\
		$q\leftarrow q_0$\\
		\Repeat{$q=q_0$}{
			Anhängen von $q$ an $L$\\
			$q\leftarrow$ minimaler Punkt im Bezug auf $<_q$ aus $Q\setminus\{q\}$
		}
		\KwRet{$L$}
	}
\end{algorithm}})
		\begin{itemize}
			\item eine Iteration der \textit{repeat}-Schleife für jeden Knoten der konvexen Hülle
			\item jedes Minimum kann in linearer Zeit bestimmt werden
			\item Gesamtlaufzeit: $\BigO(nh)$ mit $h = \#$ Knoten auf der konvexen Hülle
			\item falls $h\in o(\log n)$ ist dieses Algorithmus schneller als Graham's Scan
		\end{itemize}
\end{itemize}
\input{Tops/08/03-4-Chan.tex}