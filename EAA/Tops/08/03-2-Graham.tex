\subsection{Graham's Scan}
\begin{itemize}[itemsep=0pt]
	\item $Q$ ist die Menge von $n$ Punkten in der Ebene$q_0\in Q$
	\item $q_0\in Q$ ist der am weitesten links liegende unterste Punkt aus $Q$
	\item die Punkte werden mithilfe eines Strahls, ausgehend aus $q_0$ und rotierend entgegen des Uhrzeigersinnes, abgearbeitet
	\item der Algorithmus arbeitet die Punkte $q\in Q\setminus \{q_0\}$ in ansteigender Ordnung der Winkel zwischen $\rechts{q_0q}$ und der horizontalen Linie durch $q_0$ in positiver Richtung
	\begin{center}(in Bezug auf die Ordnung $q<_{q_0}q' \Longleftrightarrow \rechts{q_0q}$ liegt rechts von $\rechts{q_0q'}$)\end{center}
	\item wenn es eine Linie durch $q_0$ und mehrere Punkte aus $Q\setminus \{q_0\}$ gibt, wird nur der am weitesten entfernte Punkt betrachtet (nur dieser kann ein Punkt von $H(Q)$ sein)
	\item als Vorverarbeitung kann man alle Punkte außer den am weitesten entfernten Punkt löschen (bei mehreren Punkten auf einer Linie)
	\item iteratives Berechnen der konvexen Hülle von $\{q_0,\dots,q_i\},~i=3,\dots,m$
\end{itemize}
\topbreak
\up\up
\begin{itemize}[itemsep=0pt]
	\item durch die Vorverarbeitung ist $\langle q_0,q_1,q_2\rangle$ die konvexe Hülle von $\{q_0,q_1,q_2\}$
	\item $S=\langle q_0=p_1,p_2,\dots,p_{|S|}\rangle$ ist die konvexe Hülle von $\{q_0,\dots,q_{i-1}\}$
	\item beim Hinzufügen von $q_i$ muss der Innenwinkel geprüft werden (müssen wir in dem Polygon $S+q_i=\langle p_1,\dots,p_{|S|},q_i\rangle$ an der Stelle $p_{|S|}$ auf dem Weg von $q_i$ nach $p_{|S|-1}$ rechts abbiegen)\\
	ist der Innenwinkel größer als $\pi$, ist $p_{|S|}$ nicht Teil der konvexen Hülle und wird entfernt und wird nicht wieder als potentieller Teil der konvexen Hülle betrachtet
	\item iteratives Wiederholen des letzten Schrittes bis der Innenwinkel wieder kleiner als $\pi$ ist
	\item hieraus erhalten wir ein Polygon $\langle q_0=p_1,p_2,\dots,p_j,q_i\rangle$, wobei der Innenwinkel bei $p_2,\dots,p_{j-1} <\pi$, weil es schon ein konvexes Polygon war
	\item somit liegt $q_i$ links von
		\begin{enumerate}
			\item $\rechts{p_{j-1}p_j}$ durch Konstruktion
			\item $\rechts{p_{1}p_j}$ durch die Ordnung
			\item $\rechts{p_{1}p_2}$ durch die Wahl $p_1=q_0$
		\end{enumerate}
	\item \textbf{Laufzeit:} (\algo{Graham's Scan}{arg2})%TODO Algorithmus
		\begin{enumerate}
			\item Sortieren in $\BigO(n\log n)$\\
				$\Rightarrow$ alle Punkte von $Q$, die im Inneren eines Segmentes zwischen $q_0$ und einem Punkt $q\in Q$ liegen, sind direkt nach $q$ einsortiert
			\item Löschen der Punkte in (gesamt betrachtet) linearer Zeit
			\item alle übrigen Punkte werden einmal auf den Stack gepushed und höchstens einmal gepopped
			\item die for-Schleife liegt in linearer Zeit
			\item somit wird die Laufzeit vom Sortiervorgang dominiert und der Algorithmus liegt in $\BigO(n\log n)$
		\end{enumerate}
\end{itemize}