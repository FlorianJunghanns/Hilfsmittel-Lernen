\subtop{Der Algorithmus von Prim}{-1.37}
\begin{description}
	\item[Datenstruktur:]\ \\\up
		\begin{itemize}
			\item \PQ
			\item jedes Element hat einen Schlüssel, der die Priorität des Elementes abbildet
			\item kleinster Schlüssel entspricht höchster Priorität
			\item Implementation in als Heap dargestellten Bäumen oder Wäldern
			\item Laufzeit verschiedener Heaps:\\
				\begin{tabular}{c||c|c|c}
				& Binär-Heap & $d$-Heap & Fibonacci-Heap\\\hline
				\insert & $\BigO(\log n)$ & $\BigO(\log_d n)$ & $\BigO(1)$\\\hline
				\decKey & $\BigO(\log n)$&$\BigO(\log_d n)$&$\BigO(1)^{*}$\\\hline
				\exMin&$\BigO(\log n)$&$\BigO(d \log_d n)$&$\BigO(\log n)^{*}$\\\hline
				\make &$\BigO(n)$&$\BigO(n)$&$\BigO(n)$
				\end{tabular}\\\\
			$*$ amortisierte Kosten
		\end{itemize}
	\item[Operationen:]\ \\\up
		\begin{description}
			\item[\insert(item $x$, key $k$):] Einfügen eines Elementes $x$ mit Schlüssel $k$ in die \PQ
			\item[\decKey(item $x$, key $k$):] Setzen des Schlüssels von $x$ auf $k$
			\item[\exMin:] gibt das Element mit dem kleinsten Schlüssel zurück und löscht es aus der \PQ
			\item[\make:] erstellt eine \PQ~mit allen Elementen
		\end{description}
\end{description}
Während der Algorithmus läuft enthält die \PQ~alle Kanten, die nicht im blauen Baum enthalten sind. Der Schlüssel eines Knotens ist das Gewicht der leichtesten Kante $e$, die inzident zu $v$ ist und einem Knoten des blauen Baumes. Durch umhängen der Elternzeiger wird der blaue Spannbaum erzeugt.
\begin{description}
	\item[Laufzeit:] \ \\\up
		\begin{itemize}
			\item $n$ \exMin-Operationen
			\item höchstens $m+1$ \decKey-Operationen
			\item mit Fibonacci-Heaps kann der Algorithmus somit in $\BigO(m+n\log n)$ ausgeführt werden
		\end{itemize}
\end{description}
\algo{Algorithmus von Prim}{\begin{algorithm}[H]
	\SetAlgoVlined
	\SetKwProg{Fn}{Function}{}{end}
	\KwIn{Graph $G$, Kantengewichte $w:E\rightarrow\mathbb{R}$, Konten (Wurzel) $r\in V$}
	\KwOut{Knoten-Array \textit{parent}, mit $T=(V,\{\{v,parent[v]\}; v\in V\setminus \{r\}\})$ ist ein MST von $G$}
	\KwData{\PQ~$Q$}
	\BlankLine
	\Begin{
		\For{$v\in V$}{
			$Q.\insert(v,\infty)$
		}
		$Q.\decKey(r,0)$\\
		\While{$Q\neq \emptyset$}{
			$v\leftarrow Q.\exMin$\\
			\ForAll{$u$ adjazent zu $v$}{
				\If{$u$ in $Q$ und $w(\{u,v\})<key[u]$}{
					$Q.\decKey(u,w(\{u,v\}))$\\
					$parent[u] \leftarrow v$
				}
			}
		}
	}
\end{algorithm}}