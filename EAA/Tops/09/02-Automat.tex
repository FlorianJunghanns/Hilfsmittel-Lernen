\subtop{endlicher Automat-Matcher}{-1.58}
\vspace*{-0.75\baselineskip}\begin{itemize}[itemsep=-1pt]
	\item Idee: Benutzen der Informationen (in dem bisher gelesenen Text), die man durch die letzten Zeichen erhalten hat
	\item der Automat-Matcher konstruiert einen endlichen Automat $\A{P}$ für das Pattern $P$, mit den folgenden Eigenschaften:
		\begin{enumerate}
			\item $\A{P}$ liest den Text $T$ einmal
			\item $P$ kommt mit der Verschiebung $s$ in $T$ vor $\Longleftrightarrow \A{P}$ ist in einem Endzustand nachdem das Zeichen $T[s+m]$ gelesen wurde
		\end{enumerate}
	\item die Überführungsfunktion $\delta : Q \times \Sigma \rightarrow Q$ kann auf das Alphabet mit dem leeren Wort erweitert werden auf $\delta : Q\times \Sigma^{*} \rightarrow Q$ mit
		\begin{itemize}[itemsep=0pt]
			\item $\delta(q,\epsilon) = q$
			\item $\delta(q,wa) = \delta(\delta(q,w),a)$
		\end{itemize}
	\item $\A{}$ \textbf{akzeptiert} alle Wörter $w\in \Sigma^{*}$ mit $\delta(q_0,w)\in F~~(F\subseteq Q \text{ ist die Menge der Endzustände})$
	\item der String-Matching-Automat $\A{P}$ für das Pattern $P[1,\dots,m]$ ist wie folgt definiert:
		\begin{itemize}
			\item $Q=\{0,\dots,m\}$
			\item $q_0=0$
			\item $F=\{m\}$
		\end{itemize}
		sowie:\\
		\input{Pics/9_automat.pgf}\\
		wobei $\delta(q_0,[1,\dots,i])=q'$ die Länge des längsten Präfixes von $P$ ist, das ein Suffix von dem bisher gelesenen Text ist, dh.:\\
		$\begin{array}{rl}
			T[1,\dots\dots\dots\dots,i-1,i] & \text{bekannter Teil von }T\\
			P[1,\dots,q']
		\end{array}$
	\item $x\in \Sigma^{*}$ ist ein \textbf{Suffix} von $w\in\Sigma^{*}$, falls $w=yx$ für ein $y\in \Sigma^{*}$
	\item $x\in \Sigma^{*}$ ist ein \textbf{Präfix} von $w\in\Sigma^{*}$, falls $w=xy$ für ein $y\in \Sigma^{*}$
\end{itemize}
\topbreak
\up\up
\begin{itemize}[itemsep=-1pt]
	\item $\begin{array}[t]{rccl}
			\text{suf}_{P} : &\Sigma^{*}&\rightarrow & \{0,\dots,m\}\\
			& w & \mapsto & \max\{k; P[1,\dots,k]\text{ ist Suffix von }w\}
		\end{array}$
	\item $\delta(q_0,T[1,\dots,i]) = m \Longleftrightarrow P$ tritt  in $T$ auf mit der Verschiebung $i-m$
	\item die Definition der Überführungsfunktion hängt nicht vom Text $T$ ab, sondern nur vom Pattern $P$
	\item $q=\text{suf}_P(T[1,\dots,i-1])=\delta(q_0,T[1,\dots,i-1])$,\\
		$a=T[i]$,\\
		$q'=\delta(q,a)$\\
		$\Rightarrow P[1,\dots\dots,q]a$ und $P[1,\dots,q']$ sind Suffixe von $T[1,\dots,i]$:\begin{center}
			$\begin{array}{r}
				P[1,\dots,q']\\
				T[1,\dots\dots\dots,i-1,i]\\
				P[1,\dots\dots,q]a
			\end{array}$\end{center}
	\item somit kann man folgendes definieren: $\delta(q,a)=\text{suf}_P(P[1,\dots,q]a)$
	\item \textbf{Lemma:} $\delta(q_0,T[1,\dots,i])=\text{suf}_P(T[1,\dots,i]),~~i=0,\dots,n$
		\up\Proof\vspace*{-0.5\baselineskip}
			\begin{itemize}[itemsep=0pt]
				\item[] \begin{description}
						\item[IA:] $\delta(q_0,\epsilon) = q_0 = 0 = \text{suf}_P(\epsilon)$
					\end{description}
				\item[] \begin{description}
						\item[IS ($i>0$):] $\begin{array}[t]{rll}
							\delta(q_0,T[1,\dots,i]) & = & \delta(\delta(q_0,T[1,\dots,i-1],T[i])\\
							& = & \delta(\underbrace{\text{suf}_P(T[q,\dots,i-1])}_{=q},T[i])\\
							& =& \underbrace{\text{suf}_P(P[1,\dots,q]T[i])}_{=q'}\\
							& \overset{!}{=}_{(1)} & \underbrace{\text{suf}_P(T[1,\dots,i])}_{=q''}
						\end{array}$\\
						Beweis von $(1)$:
							\begin{description}
								\item[Teil 1: ($q'\leq q''$)]\ \\\up
									\begin{enumerate}
										\item $P[1,\dots,q]$ ist Suffix von $T[1,\dots,i-1]$
										\item $P[1,\dots,q']$ ist Suffix von $P[1,\dots,q]T[i]$:\\
										\hspace*{2.2cm}$\begin{array}{r}
											T[1,\dots\dots\dots\dots,i-1,i]\\
											P[1,\dots,\dots,q]T[i]\\
											P[1,\dots,q']
										\end{array}$
										\item $P[1,\dots,q']$ ist ein Suffix von $T[1,\dots,i]$\\
										\item $q''$ ist maximal mit der Eigenschaft, dass $P[1,\dots,q'']$ ein Suffix von $T[1,\dots,i]$\\
										$\Rightarrow q'\leq q''$
									\end{enumerate}
								\item[Teil 2: ($q''\leq q'$)] \ \\\up
									\begin{enumerate}
										\item Annahme: $q''>q'$
										\item da $q'$ maximal mit der Eigenschaft, dass $P[1,\dots,q']$ ein Suffix von $P[1,\dots,q]T[i]$\\
										$\Rightarrow P[1,\dots,q'']$ ist Suffix von $T[1,\dots,i]$ aber nicht von $P[1,\dots,q]T[i]$\\
										$\Rightarrow q''>q+1$ und die folgende Situation:\\
										\hspace*{2.3cm}$\begin{array}{r}
											T[1,\dots\dots\dots\dots,i-1,i]\\
											P[1,\dots\dots\dots,q'']\\
											P[1,\dots\dots,q]T[i]\\
											P[1,\dots,q']
										\end{array}$
										\item aber dann wäre $P[1,\dots,q''-1]$ ein größerer Suffix von $T[1,\dots,i-1]$ als $P[1,\dots, q]$, was der Maximalität von $q$ widerspricht
									\end{enumerate}
							\end{description}
					\end{description}
			\end{itemize}
	\item muss $(m+1)|\Sigma|$ Einträge der Überführungsfunktion berechnen
	\item \textbf{Laufzeit:}
		\begin{itemize}
			\item Naive Überführungsfunktion: $\BigO(m^3|\Sigma|)$ (kann auf $\BigO(m|\Sigma|)$ reduziert werden)\\
			\algo{Naive Überführungsfunktion}{\begin{algorithm}[H]
	\SetAlgoVlined
	\SetKwProg{Fn}{Function}{}{end}
	\KwIn{Pattern $P[1,\dots,m]$, Alphabet $\Sigma$}
	\KwOut{Überführungsfunktion $\delta$ des Automaten $\A{P}$}
	\BlankLine
	\Begin{
		\For{$q\leftarrow 0,\dots,m$}{
			\For{$a\in \Sigma$}{
				$k\leftarrow \min\{m,q+1\}$\\
				\While{$P[1,\dots,k]$ kein Suffix von $P[1,\dots,q]a$ ist}{
					$k\leftarrow k-1$
				}
				$\delta(q,a)\leftarrow k$
			}
		}
		\KwRet{$\delta$}
	}
\end{algorithm}}
			\item endlicher Automat-Matcher: $\BigO(n)$
		\end{itemize}
\end{itemize}